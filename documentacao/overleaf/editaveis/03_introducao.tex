\chapter[Introdução]{Introdução}

\textcolor{red}{Esta seção terá no máximo duas páginas para apresentar ao leitor uma breve e atualizada revisão bibliográfica sobre o tema do projeto. A Introdução mostrará ao leitor “como está o mundo atual em relação produto desenvolvido”, ou seja, citará algumas pesquisas com produtos similares, publicadas em Journals ou Teses de Doutorado.}

% \begin{equation}
%     \mathbf{F} = m \mathbf{a}
% \end{equation}

% \begin{equation}
%     x_{1,2} = \frac{-b \pm \sqrt{b^2-4ac}}{2a}
% \end{equation}

% Repare que $b^2-4ac$ pode ser negativo, gerando raízes complexas.

\textcolor{red}{Além de pesquisas científicas, é essencial que as principais legislações sobre o tema do projeto sejam citadas para atualizar o leitor. Se o grupo ou o professor orientador julgarem relevante, indicadores de mercado devem ser adicionados, como alto custo de produtos similares, baixo número de empresas concorrentes ou número estimado de consumidores finais.}

\textcolor{red}{A Introdução finalizará com 1 (um) parágrafo de justificativa. Nesse parágrafo, o grupo ressaltará o motivo para determinar que o produto proposto atenda às necessidades atual do mercado/consumidor final ou melhora algum sistema já existente, por exemplo, composto por materiais reciclados.}

\textcolor{red}{Todas as Tabelas e Figuras devem ser referenciadas ao longo do texto, já que elas são ferramentas para auxiliar no entendimento do mesmo. Use o comando ``\textsf{\textbackslash label\{\}}'' junto a Tabelas e Figuras para referencia-las. A Fig. \ref{fig:exemplo_fig} exemplifica como adicionar imagens ao texto.}

\textcolor{red}{Para citarem trabalhos, utilizem o comando ``\textsf{\textbackslash cite\{\}}''. Evitem ao máximo o uso de outros comandos, tais como o ``\textsf{\textbackslash citeonline\{\}}''.}

\begin{figure}[htpb]
\centering
\includegraphics[width=\textwidth]{figuras/fga.png}
\caption{\textcolor{red}{Exemplo de figura adicionada em \LaTeX.}}
\label{fig:exemplo_fig}
\end{figure}

% \chapter*[Introdução]{Introdução}
% \addcontentsline{toc}{chapter}{Introdução}

% Este documento apresenta considerações gerais e preliminares relacionadas 
% à redação de relatórios de Projeto de Graduação da Faculdade UnB Gama 
% (FGA). São abordados os diferentes aspectos sobre a estrutura do trabalho, 
% uso de programas de auxilio a edição, tiragem de cópias, encadernação, etc.

% Este template é uma adaptação do ABNTeX2\footnote{\url{https://github.com/abntex/abntex2/}}.
